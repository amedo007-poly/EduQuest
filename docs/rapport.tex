\documentclass[12pt,a4paper]{article}
\usepackage[utf8]{inputenc}
\usepackage[T1]{fontenc}
\usepackage[french]{babel}
\usepackage{graphicx}
\usepackage{xcolor}
\usepackage{tikz}
\usepackage{pgfplots}
\usepackage{fancyhdr}
\usepackage{geometry}
\usepackage{hyperref}
\usepackage{listings}
\usepackage{booktabs}
\usepackage{float}
\usepackage{fontawesome5}
\usepackage{tcolorbox}
\usepackage{enumitem}
\usepackage{amsmath}
\usepackage{multicol}
\usepackage{wrapfig}

\usetikzlibrary{shapes.geometric, arrows.meta, positioning, calc, shadows, decorations.pathreplacing}
\pgfplotsset{compat=1.18}

\geometry{margin=2.5cm}

% Colors - Duolingo inspired
\definecolor{primary}{HTML}{58CC02}
\definecolor{secondary}{HTML}{1CB0F6}
\definecolor{accent}{HTML}{FF9600}
\definecolor{danger}{HTML}{FF4B4B}
\definecolor{purple}{HTML}{CE82FF}
\definecolor{dark}{HTML}{3C3C3C}
\definecolor{codebg}{HTML}{F7F7F7}
\definecolor{bigtech}{HTML}{EA4335}
\definecolor{opensource}{HTML}{58CC02}

% Header/Footer
\pagestyle{fancy}
\fancyhf{}
\fancyhead[L]{\textcolor{primary}{\textbf{EduQuest}} \textcolor{gray}{| Village Numérique Résistant}}
\fancyhead[R]{\textcolor{gray}{Nuit de l'Info 2025}}
\fancyfoot[C]{\thepage}
\renewcommand{\headrulewidth}{2pt}
\renewcommand{\headrule}{\hbox to\headwidth{\color{primary}\leaders\hrule height \headrulewidth\hfill}}

% Code style
\lstset{
    backgroundcolor=\color{codebg},
    basicstyle=\ttfamily\small,
    breaklines=true,
    frame=single,
    rulecolor=\color{primary},
    keywordstyle=\color{secondary}\bfseries,
    commentstyle=\color{primary},
    stringstyle=\color{accent},
}

% Custom boxes
\newtcolorbox{featurebox}[1]{
    colback=primary!5,
    colframe=primary,
    fonttitle=\bfseries,
    title=#1,
    rounded corners,
    boxrule=2pt
}

\newtcolorbox{warningbox}[1]{
    colback=danger!5,
    colframe=danger,
    fonttitle=\bfseries,
    title=#1,
    rounded corners,
    boxrule=2pt
}

\newtcolorbox{infobox}{
    colback=secondary!10,
    colframe=secondary,
    rounded corners,
    boxrule=2pt
}

\newtcolorbox{challengebox}[1]{
    colback=purple!10,
    colframe=purple,
    fonttitle=\bfseries,
    title=#1,
    rounded corners,
    boxrule=2pt
}

\begin{document}

% Title Page
\begin{titlepage}
    \centering
    \vspace*{1cm}
    
    % Logo with Duolingo-style owl inspiration
    \begin{tikzpicture}
        \node[circle, fill=primary, minimum size=4cm, drop shadow={shadow xshift=3pt, shadow yshift=-3pt}] (logo) {};
        \node at (logo) {\textcolor{white}{\fontsize{50}{60}\selectfont\faGraduationCap}};
        \node[circle, fill=accent, minimum size=1cm] at (1.2,1.2) {};
        \node at (1.2,1.2) {\textcolor{white}{\faRobot}};
    \end{tikzpicture}
    
    \vspace{0.8cm}
    
    {\Huge\bfseries\textcolor{primary}{EduQuest}}\\[0.3cm]
    {\Large\textcolor{secondary}{Le Village Numérique Résistant}}\\[0.5cm]
    {\large\textcolor{dark}{Plateforme d'Apprentissage Gamifiée, Open Source \& IA}}\\[0.3cm]
    
    \begin{tikzpicture}
        \node[fill=accent!20, rounded corners=5pt, inner sep=8pt] {
            \textcolor{accent}{\textbf{« L'école reprend le contrôle face aux Big Tech »}}
        };
    \end{tikzpicture}
    
    \vspace{1.5cm}
    
    \begin{tikzpicture}
        \draw[primary, very thick, rounded corners=15pt, fill=primary!5] (0,0) rectangle (13,4);
        \node[align=center] at (6.5,2) {
            {\Large\textbf{Nuit de l'Info 2025}}\\[0.4cm]
            {\large Documentation Technique \& Conceptuelle}\\[0.3cm]
            {\normalsize Défis: MiniMind (AI4GOOD) • Chat'bruti (Viveris) • Green IT}
        };
    \end{tikzpicture}
    
    \vspace{1.5cm}
    
    % Stats boxes
    \begin{tikzpicture}
        \node[draw=primary, fill=primary!10, rounded corners, minimum width=2.8cm, minimum height=1.8cm] at (0,0) {
            \begin{tabular}{c}
                \Large\textcolor{primary}{\textbf{26}}\\
                \small Quiz
            \end{tabular}
        };
        \node[draw=secondary, fill=secondary!10, rounded corners, minimum width=2.8cm, minimum height=1.8cm] at (3.5,0) {
            \begin{tabular}{c}
                \Large\textcolor{secondary}{\textbf{12}}\\
                \small Matières
            \end{tabular}
        };
        \node[draw=accent, fill=accent!10, rounded corners, minimum width=2.8cm, minimum height=1.8cm] at (7,0) {
            \begin{tabular}{c}
                \Large\textcolor{accent}{\textbf{IA}}\\
                \small DeepSeek
            \end{tabular}
        };
        \node[draw=purple, fill=purple!10, rounded corners, minimum width=2.8cm, minimum height=1.8cm] at (10.5,0) {
            \begin{tabular}{c}
                \Large\textcolor{purple}{\textbf{100\%}}\\
                \small Open Source
            \end{tabular}
        };
    \end{tikzpicture}
    
    \vfill
    
    {\large\textbf{Équipe:} SuperAmedo | Polytechnique Mauritanie}\\[0.3cm]
    {\large\textbf{Date:} 5-6 Décembre 2025}\\[0.3cm]
    {\normalsize\textbf{Stack:} React • Flask • Docker • DeepSeek API • SQLite}\\[0.3cm]
    {\small\textcolor{gray}{GitHub: github.com/amedo007-poly/EduQuest}}
    
\end{titlepage}

% Table of Contents
\tableofcontents
\newpage

% ============ PAGE 2 ============
\section{Introduction et Vision}

\subsection{Le Problème: La Domination des Big Tech}

\begin{warningbox}{\faExclamationTriangle\ Le Constat Alarmant}
Aujourd'hui, l'éducation numérique est dominée par les \textbf{GAFAM} (Google, Apple, Facebook, Amazon, Microsoft). Les écoles dépendent de:
\begin{itemize}
    \item Google Classroom, Microsoft Teams → \textcolor{danger}{données des élèves collectées}
    \item Applications propriétaires → \textcolor{danger}{coûts de licence, vendor lock-in}
    \item Algorithmes opaques → \textcolor{danger}{pas de contrôle sur l'apprentissage}
\end{itemize}
\end{warningbox}

\subsection{Notre Réponse: Le Village Numérique Résistant}

\begin{featurebox}{\faShieldAlt\ EduQuest - La Solution Open Source}
\textbf{EduQuest} est une plateforme d'apprentissage gamifiée, inspirée de \textbf{Duolingo} mais \textbf{100\% open source}, que les écoles peuvent héberger elles-mêmes.

\vspace{0.3cm}
\textit{« Nous proposons une alternative éthique où l'IA sert les enseignants et les élèves, pas les publicitaires. Les données restent dans l'établissement – pas chez les Big Tech. »}
\end{featurebox}

\subsection{Les 3 Piliers de Notre Vision}

\begin{center}
\begin{tikzpicture}[
    pillar/.style={rectangle, rounded corners=10pt, minimum width=4cm, minimum height=3cm, align=center, font=\small}
]
    % Pillar 1 - AI
    \node[pillar, fill=secondary!20, draw=secondary, line width=2pt] (ai) at (0,0) {
        \textcolor{secondary}{\Huge\faRobot}\\[0.3cm]
        \textbf{Section 1}\\
        \textbf{IA Adaptative}\\[0.2cm]
        Personnalisation\\
        Recommandations\\
        Chatbot Tuteur
    };
    
    % Pillar 2 - Gamification
    \node[pillar, fill=accent!20, draw=accent, line width=2pt] (game) at (5,0) {
        \textcolor{accent}{\Huge\faGamepad}\\[0.3cm]
        \textbf{Section 2}\\
        \textbf{Gamification}\\[0.2cm]
        Score \& Niveaux\\
        Streaks Quotidiens\\
        Classement
    };
    
    % Pillar 3 - Open Source
    \node[pillar, fill=primary!20, draw=primary, line width=2pt] (open) at (10,0) {
        \textcolor{primary}{\Huge\faCodeBranch}\\[0.3cm]
        \textbf{Section 3}\\
        \textbf{Open Source}\\[0.2cm]
        Auto-hébergeable\\
        Sans trackers\\
        Données privées
    };
\end{tikzpicture}
\end{center}

\subsection{Défis Nuit de l'Info Ciblés}

\begin{challengebox}{\faTrophy\ Défis Auxquels Nous Répondons}
\begin{enumerate}
    \item \textbf{MiniMind (AI4GOOD - 200\$):} Prototype IA pédagogique interactif pour les jeunes
    \item \textbf{Chat'bruti (Viveris - 600€):} Chatbot avec personnalité (Socratouille intégré)
    \item \textbf{Green IT (N2ITLSE):} Application éco-conçue, légère et optimisée
    \item \textbf{Défi National:} Réponse à "Comment l'école résiste aux Big Tech?"
\end{enumerate}
\end{challengebox}

\newpage

% ============ PAGE 3 ============
\section{Architecture Technique}

\subsection{Vue d'Ensemble du Système}

\begin{center}
\begin{tikzpicture}[
    node distance=1.5cm,
    box/.style={rectangle, draw=#1, fill=#1!15, rounded corners=8pt, minimum width=3.5cm, minimum height=1.8cm, align=center, font=\small\bfseries, line width=1.5pt},
    arrow/.style={-Stealth, thick, #1}
]
    % User
    \node[box=dark] (user) {\faUser\ \textbf{Élève/Prof}\\Navigateur Web};
    
    % Frontend
    \node[box=secondary, right=2.5cm of user] (frontend) {\faReact\ \textbf{React}\\Frontend SPA};
    
    % Backend
    \node[box=primary, right=2.5cm of frontend] (backend) {\faPython\ \textbf{Flask}\\API REST};
    
    % Database
    \node[box=accent, below=1.5cm of backend] (db) {\faDatabase\ \textbf{SQLite}\\Base de données};
    
    % AI
    \node[box=purple, above=1.5cm of backend] (ai) {\faRobot\ \textbf{DeepSeek}\\IA Tuteur (API)};
    
    % Docker container
    \node[draw=dark, dashed, rounded corners=15pt, fit=(frontend)(backend)(db), inner sep=20pt, label={[font=\bfseries]above:\faDocker\ Docker Container}] (docker) {};
    
    % Arrows
    \draw[arrow=dark] (user) -- node[above, font=\scriptsize] {HTTPS} (frontend);
    \draw[arrow=secondary] (frontend) -- node[above, font=\scriptsize] {API REST} (backend);
    \draw[arrow=primary] (backend) -- (db);
    \draw[arrow=purple] (backend) -- node[right, font=\scriptsize] {OpenRouter} (ai);
    
\end{tikzpicture}
\end{center}

\subsection{Stack Technologique - 100\% Open Source}

\begin{table}[H]
\centering
\renewcommand{\arraystretch}{1.3}
\begin{tabular}{llll}
\toprule
\textbf{Composant} & \textbf{Technologie} & \textbf{Licence} & \textbf{Alternative Big Tech}\\
\midrule
\rowcolor{secondary!10} Frontend & React.js 18 & MIT & Google Angular \\
Backend & Flask 3.0 & BSD & AWS Lambda \\
\rowcolor{secondary!10} Base de données & SQLite 3 & Public Domain & Cloud Firestore \\
Authentification & JWT + bcrypt & MIT & Auth0, Firebase \\
\rowcolor{secondary!10} IA/LLM & DeepSeek (API libre) & — & ChatGPT (payant) \\
Conteneurisation & Docker & Apache 2.0 & — \\
\rowcolor{secondary!10} Hébergement & Self-hosted & — & AWS/GCP/Azure \\
\bottomrule
\end{tabular}
\caption{Stack 100\% libre vs alternatives propriétaires}
\end{table}

\subsection{Structure du Projet}

\begin{lstlisting}[language=bash, caption=Architecture modulaire]
EduQuest/
├── backend/                  # API Flask
│   ├── app.py               # Point d'entrée
│   ├── models.py            # User, Quiz, Attempt (SQLAlchemy)
│   ├── seed_data.py         # 26 quiz, 10 utilisateurs démo
│   ├── routes/
│   │   ├── auth.py          # JWT login/register
│   │   ├── quiz.py          # CRUD quiz + scoring
│   │   ├── leaderboard.py   # Classement temps réel
│   │   └── ai.py            # Tuteur DeepSeek
│   └── services/
│       ├── ai_tutor.py      # Intégration OpenRouter
│       └── scoring.py       # Algorithme XP + streak
├── frontend/                 # React SPA
│   ├── src/
│   │   ├── components/      # Dashboard, Quiz, Leaderboard
│   │   ├── App.js           # Router + Auth context
│   │   └── index.js
│   └── package.json
├── docker-compose.yml        # Déploiement one-click
└── docs/rapport.tex          # Cette documentation
\end{lstlisting}

\newpage

% ============ PAGE 4 ============
\section{Section 1: IA Adaptative et Personnalisée}

\subsection{Profil Apprenant Dynamique}

Chaque utilisateur dispose d'un profil qui évolue en temps réel:

\begin{center}
\begin{tikzpicture}
    \node[draw=secondary, fill=secondary!10, rounded corners=15pt, minimum width=12cm, minimum height=4cm] (profile) {};
    \node[anchor=north west] at (-5.5,1.7) {\Large\textbf{Profil Utilisateur}};
    
    % Stats
    \node at (-3.5,0.5) {\faUser\ \textbf{Niveau:} 5};
    \node at (-3.5,-0.2) {\faStar\ \textbf{XP:} 2,500 pts};
    \node at (-3.5,-0.9) {\faFire\ \textbf{Streak:} 12 jours};
    
    % Progress bar
    \draw[gray!30, line width=8pt, rounded corners=3pt] (0,0.5) -- (4.5,0.5);
    \draw[primary, line width=8pt, rounded corners=3pt] (0,0.5) -- (3.5,0.5);
    \node[right] at (4.7,0.5) {\textbf{78\%}};
    \node[above] at (2.25,0.8) {\small Progression niveau 6};
    
    % Interests
    \node at (2.25,-0.5) {\textbf{Points forts:} Maths, Sciences};
    \node at (2.25,-1.1) {\textbf{À améliorer:} Histoire, Géographie};
\end{tikzpicture}
\end{center}

\subsection{Système de Recommandation}

\begin{infobox}
L'algorithme de recommandation analyse le profil de l'élève pour suggérer:
\begin{itemize}
    \item Le \textbf{prochain quiz} adapté à son niveau
    \item Les \textbf{matières à réviser} (basé sur les erreurs)
    \item La \textbf{difficulté optimale} (ni trop facile, ni trop dur)
\end{itemize}
\end{infobox}

\subsection{EduBot - Tuteur IA Personnel (Défi MiniMind)}

\begin{center}
\begin{tikzpicture}[
    bubble/.style={rectangle, rounded corners=10pt, minimum width=5cm, align=left, font=\small}
]
    % Bot avatar
    \node[circle, fill=secondary, minimum size=1.5cm] (bot) at (-4,0) {\textcolor{white}{\Large\faRobot}};
    \node[below=0.1cm of bot] {\textbf{EduBot}};
    
    % Chat bubbles
    \node[bubble, fill=gray!20, anchor=west] at (-2.5,1.5) {
        \faUser\ « Je ne comprends pas les fractions... »
    };
    
    \node[bubble, fill=secondary!20, anchor=west] at (-2.5,0) {
        \faRobot\ « Imagine une pizza coupée en 4 parts.\\
        Si tu en manges 1, tu as mangé $\frac{1}{4}$.\\
        C'est 1 part sur 4 au total! 🍕 »
    };
    
    \node[bubble, fill=gray!20, anchor=west] at (-2.5,-1.5) {
        \faUser\ « Ah je comprends mieux! »
    };
\end{tikzpicture}
\end{center}

\subsubsection{Configuration IA}

\begin{table}[H]
\centering
\begin{tabular}{ll}
\toprule
\textbf{Paramètre} & \textbf{Valeur}\\
\midrule
Provider & OpenRouter (API ouverte) \\
Modèle & DeepSeek R1T2 Chimera \\
Coût & \textcolor{primary}{\textbf{Gratuit}} (Free Tier) \\
Max Tokens & 500 \\
Température & 0.7 (créatif mais cohérent) \\
Prompt System & Pédagogue bienveillant \\
\bottomrule
\end{tabular}
\end{table}

\subsection{Prompt Engineering Pédagogique}

\begin{lstlisting}[caption=Prompt système du tuteur]
Tu es EduBot, un tuteur IA bienveillant specialise dans 
l'education. Tu aides les eleves a comprendre leurs erreurs 
sans donner directement les reponses. 

Regles:
- Utilise des analogies simples (pizza, bonbons, sport)
- Pose des questions pour guider la reflexion
- Encourage toujours l'eleve
- Reponds en francais
- Sois patient et positif
\end{lstlisting}

\newpage

% ============ PAGE 5 ============
\section{Section 2: Gamification à la Duolingo}

\subsection{Le Modèle Duolingo - Gratuit et Open Source}

\begin{center}
\begin{tikzpicture}
    % Comparison table visual
    \node[draw=danger, fill=danger!10, rounded corners, minimum width=5.5cm, minimum height=3cm, align=center] at (-3.5,0) {
        \textbf{\textcolor{danger}{Duolingo (Propriétaire)}}\\[0.3cm]
        \faTimesCircle\ Données vendues\\
        \faTimesCircle\ Publicités intrusives\\
        \faTimesCircle\ Premium payant (7€/mois)\\
        \faTimesCircle\ Code fermé
    };
    
    \node[draw=primary, fill=primary!10, rounded corners, minimum width=5.5cm, minimum height=3cm, align=center] at (3.5,0) {
        \textbf{\textcolor{primary}{EduQuest (Open Source)}}\\[0.3cm]
        \faCheckCircle\ Données privées\\
        \faCheckCircle\ Zéro publicité\\
        \faCheckCircle\ 100\% gratuit\\
        \faCheckCircle\ Code ouvert (GitHub)
    };
    
    \node at (0,0) {\Huge\textbf{VS}};
\end{tikzpicture}
\end{center}

\subsection{Système de Scoring et Niveaux}

\begin{featurebox}{\faStar\ Algorithme de Points d'Expérience}
\begin{equation}
\text{XP Gagné} = \text{Bonnes Réponses} \times 10 \times \text{Multiplicateur Streak}
\end{equation}

\begin{equation}
\text{Multiplicateur} = 1 + \min(0.5, \text{jours\_streak} \times 0.05)
\end{equation}
\end{featurebox}

\begin{center}
\begin{tikzpicture}
\begin{axis}[
    title={\textbf{Évolution du Multiplicateur de Streak}},
    xlabel={Jours consécutifs},
    ylabel={Multiplicateur XP},
    xmin=0, xmax=15,
    ymin=1, ymax=1.6,
    grid=major,
    width=11cm,
    height=6cm,
    legend pos=south east,
]
\addplot[domain=0:10, samples=50, thick, primary, mark=*] {1 + x*0.05};
\addplot[domain=10:15, samples=20, thick, accent, dashed] {1.5};
\legend{Croissance, Max (×1.5)}
\end{axis}
\end{tikzpicture}
\end{center}

\subsection{Streak System - Motivation Quotidienne}

\begin{center}
\begin{tikzpicture}
    % Week display like Duolingo
    \foreach \x/\day/\col in {0/L/primary, 1.5/M/primary, 3/M/primary, 4.5/J/primary, 6/V/primary, 7.5/S/gray, 9/D/gray} {
        \node[circle, fill=\col!30, draw=\col, line width=2pt, minimum size=1.2cm] at (\x,0) {\textbf{\day}};
    }
    \node[above] at (4.5,1) {\textbf{\faFire\ Streak: 5 jours consécutifs!}};
    
    % Fire icons for completed days
    \foreach \x in {0, 1.5, 3, 4.5, 6} {
        \node[text=accent] at (\x,-0.05) {\small\faFire};
    }
\end{tikzpicture}
\end{center}

\subsection{Règle de Reset (Challenge)}

\begin{warningbox}{\faExclamationTriangle\ Règle de Motivation}
\textbf{3 échecs consécutifs = Score remis à 0!}

Cette règle encourage les élèves à:
\begin{itemize}
    \item Se concentrer avant de répondre
    \item Réviser les matières difficiles
    \item Demander de l'aide au tuteur IA
\end{itemize}
\end{warningbox}

\subsection{Classement et Compétition Saine}

\begin{center}
\begin{tikzpicture}
    % Leaderboard
    \node[draw=dark, fill=dark!5, rounded corners=10pt, minimum width=10cm, minimum height=4cm] {};
    \node[above] at (0,1.7) {\Large\textbf{\faTrophy\ Classement Global}};
    
    % Entries
    \node[fill=yellow!30, rounded corners, minimum width=8cm] at (0,0.8) {
        \textbf{1.} \faMedal\ MathGenius \hfill \textbf{2,500 XP} \hfill Niveau 5
    };
    \node[fill=gray!20, rounded corners, minimum width=8cm] at (0,0.1) {
        \textbf{2.} \faMedal\ ScienceQueen \hfill \textbf{2,200 XP} \hfill Niveau 5
    };
    \node[fill=orange!20, rounded corners, minimum width=8cm] at (0,-0.6) {
        \textbf{3.} \faMedal\ HistoryBuff \hfill \textbf{1,800 XP} \hfill Niveau 4
    };
\end{tikzpicture}
\end{center}

\newpage

% ============ PAGE 6 ============
\section{Section 3: Open Source \& Résistance Numérique}

\subsection{Pourquoi Résister aux Big Tech?}

\begin{center}
\begin{tikzpicture}[
    node distance=0.8cm
]
    \node[draw=danger, fill=danger!10, rounded corners=10pt, minimum width=13cm, minimum height=5cm] {};
    \node[anchor=north, font=\Large\bfseries] at (0,2.2) {\textcolor{danger}{\faExclamationTriangle\ Les Dangers de la Dépendance Big Tech}};
    
    \node[align=left] at (0,0.5) {
        \begin{tabular}{ll}
            \faGoogle\ \textbf{Google Classroom} & Collecte les données d'apprentissage des enfants \\
            \faMicrosoft\ \textbf{Microsoft Teams} & Dépendance au cloud payant (Office 365) \\
            \faApple\ \textbf{Apple iPad} & Matériel propriétaire coûteux \\
            \faAmazon\ \textbf{AWS} & Coûts explosent avec l'usage \\
        \end{tabular}
    };
    
    \node[align=center, font=\itshape] at (0,-1.5) {
        « Les écoles deviennent des \textbf{clients captifs} des GAFAM. »
    };
\end{tikzpicture}
\end{center}

\subsection{Notre Réponse: Souveraineté Numérique}

\begin{featurebox}{\faShieldAlt\ Les 5 Principes d'EduQuest}
\begin{enumerate}
    \item \textbf{Auto-hébergeable:} L'école peut installer la plateforme sur son propre serveur
    \item \textbf{Données locales:} SQLite = tout reste sur le serveur de l'école
    \item \textbf{Zéro tracking:} Pas de cookies publicitaires, pas d'analytics tierces
    \item \textbf{IA éthique:} Modèle choisi par l'école (DeepSeek, Mistral, Llama...)
    \item \textbf{Gratuit à vie:} Licence MIT, aucun coût caché
\end{enumerate}
\end{featurebox}

\subsection{Éco-conception (Défi Green IT)}

\begin{center}
\begin{tikzpicture}
\begin{axis}[
    title={\textbf{Empreinte Carbone Comparée}},
    ybar,
    symbolic x coords={EduQuest, Google Classroom, Microsoft Teams},
    xtick=data,
    ylabel={kg CO2/an/utilisateur},
    ymin=0, ymax=25,
    bar width=1.5cm,
    nodes near coords,
    nodes near coords style={font=\bfseries},
    width=12cm,
    height=7cm,
]
\addplot[fill=primary!70] coordinates {
    (EduQuest, 2)
    (Google Classroom, 18)
    (Microsoft Teams, 22)
};
\end{axis}
\end{tikzpicture}
\end{center}

\subsubsection{Optimisations Green IT}

\begin{table}[H]
\centering
\begin{tabular}{lll}
\toprule
\textbf{Métrique} & \textbf{EduQuest} & \textbf{Objectif Green IT}\\
\midrule
Taille DOM & < 500 éléments & \textcolor{primary}{\faCheck} \\
Poids page & < 500 KB & \textcolor{primary}{\faCheck} \\
Requêtes HTTP & < 20 & \textcolor{primary}{\faCheck} \\
Images & WebP compressé & \textcolor{primary}{\faCheck} \\
JavaScript & Minifié + lazy load & \textcolor{primary}{\faCheck} \\
\bottomrule
\end{tabular}
\caption{Conformité aux critères Green IT de N2ITLSE}
\end{table}

\subsection{Les 12 Matières Couvertes}

\begin{center}
\begin{tikzpicture}[
    subject/.style={rectangle, rounded corners=5pt, minimum width=2.5cm, minimum height=0.9cm, align=center, font=\small\bfseries}
]
    % Row 1
    \node[subject, fill=primary!25] at (0,0) {\faCalculator\ Maths};
    \node[subject, fill=secondary!25] at (3,0) {\faBook\ Français};
    \node[subject, fill=accent!25] at (6,0) {\faFlask\ Sciences};
    \node[subject, fill=purple!25] at (9,0) {\faLandmark\ Histoire};
    
    % Row 2
    \node[subject, fill=primary!25] at (0,-1.2) {\faGlobe\ Géographie};
    \node[subject, fill=secondary!25] at (3,-1.2) {\faCode\ Programmation};
    \node[subject, fill=accent!25] at (6,-1.2) {\faLanguage\ Anglais};
    \node[subject, fill=purple!25] at (9,-1.2) {\faMusic\ Musique};
    
    % Row 3
    \node[subject, fill=primary!25] at (1.5,-2.4) {\faFutbol\ Sports};
    \node[subject, fill=secondary!25] at (4.5,-2.4) {\faPaintBrush\ Art};
    \node[subject, fill=accent!25] at (7.5,-2.4) {\faLightbulb\ Philosophie};
    \node[subject, fill=purple!25] at (10.5,-2.4) {\faChartLine\ Économie};
\end{tikzpicture}
\end{center}

\newpage

% ============ PAGE 7 ============
\section{Déploiement et Utilisation}

\subsection{Installation en Une Commande}

\begin{lstlisting}[language=bash, caption=Déploiement Docker instantané]
# 1. Cloner le repository
git clone https://github.com/amedo007-poly/EduQuest.git
cd EduQuest

# 2. Lancer toute la stack
docker-compose up -d

# C'est tout! L'application est prete.
# Frontend: http://localhost:3000
# Backend API: http://localhost:5000/api
\end{lstlisting}

\subsection{Comptes de Démonstration}

\begin{table}[H]
\centering
\renewcommand{\arraystretch}{1.2}
\begin{tabular}{llcccc}
\toprule
\textbf{Nom} & \textbf{Email} & \textbf{Mot de passe} & \textbf{Niveau} & \textbf{XP} & \textbf{Streak}\\
\midrule
\rowcolor{yellow!15} MathGenius & math@demo.com & demo123 & 5 & 2,500 & 12 \\
ScienceQueen & science@demo.com & demo123 & 5 & 2,200 & 10 \\
\rowcolor{yellow!15} HistoryBuff & history@demo.com & demo123 & 4 & 1,800 & 15 \\
CodeMaster & code@demo.com & demo123 & 4 & 1,650 & 8 \\
\bottomrule
\end{tabular}
\caption{10 utilisateurs démo pré-créés pour tester le classement}
\end{table}

\subsection{API Endpoints}

\begin{table}[H]
\centering
\begin{tabular}{lll}
\toprule
\textbf{Méthode} & \textbf{Endpoint} & \textbf{Description}\\
\midrule
POST & /api/auth/register & Inscription utilisateur \\
POST & /api/auth/login & Connexion (retourne JWT) \\
GET & /api/quiz/ & Liste des quiz disponibles \\
GET & /api/quiz/<id> & Détails d'un quiz \\
POST & /api/quiz/<id>/submit & Soumettre réponses \\
GET & /api/leaderboard/ & Classement global \\
POST & /api/ai/chat & Conversation avec EduBot \\
GET & /api/user/profile & Profil utilisateur \\
\bottomrule
\end{tabular}
\caption{API REST complète}
\end{table}

\section{Conclusion et Perspectives}

\subsection{Ce Que Nous Avons Accompli}

\begin{center}
\begin{tikzpicture}
    \node[draw=primary, fill=primary!10, rounded corners=15pt, minimum width=13cm, minimum height=4.5cm] {};
    
    \node[align=left] at (0,0.8) {
        \textcolor{primary}{\faCheckCircle} \textbf{Plateforme complète} avec 26 quiz et 12 matières\\[0.2cm]
        \textcolor{primary}{\faCheckCircle} \textbf{Gamification Duolingo-style} : XP, streaks, classement\\[0.2cm]
        \textcolor{primary}{\faCheckCircle} \textbf{Tuteur IA} EduBot (DeepSeek) intégré\\[0.2cm]
        \textcolor{primary}{\faCheckCircle} \textbf{100\% Open Source} et auto-hébergeable\\[0.2cm]
        \textcolor{primary}{\faCheckCircle} \textbf{Docker} : déploiement en 1 commande
    };
\end{tikzpicture}
\end{center}

\subsection{Pitch Final}

\begin{center}
\begin{tikzpicture}
    \node[draw=secondary, fill=secondary, text=white, rounded corners=10pt, minimum width=13cm, minimum height=2cm, font=\large, align=center] {
        \textbf{« EduQuest : L'école reprend le contrôle.}\\
        \textbf{Apprentissage gamifié, IA éthique, 100\% libre. »}
    };
\end{tikzpicture}
\end{center}

\subsection{Liens Importants}

\begin{center}
\begin{tikzpicture}
    \node[draw=dark, fill=dark!5, rounded corners=10pt, minimum width=10cm, minimum height=2.5cm] {};
    \node at (0,0.5) {\faGithub\ \textbf{GitHub:} \texttt{github.com/amedo007-poly/EduQuest}};
    \node at (0,-0.1) {\faDocker\ \textbf{Docker Hub:} Image disponible};
    \node at (0,-0.7) {\faRobot\ \textbf{IA:} DeepSeek via OpenRouter (gratuit)};
\end{tikzpicture}
\end{center}

\vspace{1cm}

\begin{center}
\begin{tikzpicture}
    \node[circle, fill=primary, minimum size=2cm, drop shadow] (logo) {\textcolor{white}{\Huge\faGraduationCap}};
    \node[below=0.5cm of logo, font=\Large\bfseries] {\textcolor{primary}{Apprendre n'a jamais été aussi libre!}};
\end{tikzpicture}
\end{center}

\end{document}
